\section{Data management and data annotation - use case of Cern-VM in integrative structural biology}

Structural biology is part of molecular biology focusing on determining structure of macromolecules inside living cells and cell membranes. Integrative structural biology aims to leverage combined usage of tools, techniques and methods used to determine the structure including nuclear magnetic resonance (NMR), X-Ray crystalography, cryo electron microscopy and others. Each method has it's advantages and disadvantages in the terms of availability, sample preparation, resolution. 

West-Life project ambition is facilitate integrative approach using multiple techniques mentioned above. As there are already lot of software tools to process data produced by the techniques above, the challenge is to integrate them together in a way they can be used by experts in one technique but not experts in other techniques. One product of the West-Life project is a data management service - virtual folder. It delivers a uniform way to integrate scattered data from different storage providers. Another product is a virtual machine, which may allow to launch specific software tools to process user's data in virtual folder. CernVM with option to be launched with graphical user interface is used as a basic template to contextualize virtual machine with additional structural biology software suites such as CCP4, Scipion and others. CernVM-FS is used to distribute updates of structural biology software suites as well as West-Life specific services - virtual folder and newly repository. The virtual machine templates are available in EGI's APP DB as well as within STFC cloud computing infrastructure.